\documentclass{article}
\usepackage{fancyhdr,booktabs}
\usepackage{amsmath}
\usepackage{float}
\usepackage{graphicx}
\usepackage{indentfirst}
\usepackage{geometry}
\usepackage{citesort}

\begin{document}

\section{Numerical Methods}
\section{Confusion problem}

Given a waveform under specetiem with non-zero deformation parameter, we need to decide which waveform under Kerr specetime is most similar to it.

If we restrict ourselves to equatorial motion, set the initial $t$ and $\phi$ to 0 taking advantage of symmetry and set initial $r=r_{max}$ imposing the phase to match, orbital eccentricity $e$, semilatus rectum $p$, BH mass $M$ and BH spin $a$ are the parameters that determine the motion.

According to Ref. \cite{sameOmg}, orbits with same orbital frequency $\omega_r$ and $\omega_\phi$ can generate most similar gravitational waveforms. Here we check this result by lookinf at overlaps between waveforms with ($\delta_1,\, a,\, M,\, e,\, p$) = (0.2, 0.5, , 0.5, 6). First we look at overlap distribution on a relatively large range of (e, p). Then we search near ($e_{Kerr}$, $p_{Kerr}$) with same orbital frequency, as shown in Fig. \ref{overlapdist}. 

Note that the difference between equating $\omega^{(t)}$, orbital frequency with respect to coordiante time, and $\omega^{(\tau)}$, orbital frequency with respect to proper time, can be significant. From Fig. \ref{overlapdist} it is explicit that the same orbital frequency with respect to t can result in almost the largest overlap while the same orbital frequency with respect to $\tau$ cannot. In Kerr spacetime, the expression for $\omega^{(t)}$ and $\omega^{(\tau)}$ are given in \cite{tOmg} and \cite{tauOmg} 

\begin{figure}[]
	\centering
	\includegraphics[width=16cm]{OLdist.png}
	\includegraphics[width=16cm]{OLdist2.png}
	
	\caption{Distribution of overlap between waveform of ($\delta_1,\, a,\, M,\, e,\, p$) = (0.2, 0.5, , 0.5, 6) and waveforms of ($\delta_1,\, a,\, M,\, e,\, p$) =(0.2, 0.5, , $e_{Kerr}$, $p_{Kerr}$) on ($e_{Kerr}$, $p_{Kerr}$) plane. The original data are both 50*50 grid. Red cross mark: same $\omega^{(t)}$ at ($e_{Kerr}$, $p_{Kerr}$) = (0.409248, 6.481170), overlap is 0.8731. Black plus mark: same $\omega^{(\tau)}$ at  ($e_{Kerr}$, $p_{Kerr}$) = (0.411495, 6.482549), overlap is 0.0507 }
	\label{overlapdist}
\end{figure}

\begin{figure}[!htb]
	\centering
	\includegraphics[width=16cm]{krz_kerr_wave.png}
	
	\caption{Comparison between waveforms of $h_+$ with respect to retarded time in units of central black hole mass $M$. The black solid line is the waveform under $\delta_1=0.2$, e=0.5,p=6. The red dashed line is the waveform under $\delta_1=0$ and e, p adapted so that the orbital frequencies with respect to t $\omega^{(t)}_r$ and $\omega^{(t)}_\phi$ are the same as that of the orbit under d1=0.2, e=0.5, p=6. The green dotted line is the waveform under $\delta_1 =0$ and e, p adopted so that $\omega^{(\tau)}$s are the same . The spin of the central black hole is 0.5M.}
	\label{kkwave}
\end{figure}	
	
Therefore we regard waveforms in Kerr spacetime with same orbital frequencies as best matches to waveforms in non-Kerr space time under KRZ parametrization.

Fig. shows the overlap distrobution...

As Fig. suggest, the confusion problem still exists in KRZ parametrization. The deformation parameter $\delta_1$ is kind of degenerated with   in Kerr spacetime. This resulted can also be found by looking at covariance matrix as discussed in next section
\section{Constraints on deformation parameter by future LISA task}

\bibliography{citation}
\bibliographystyle{plain}
\end{document}